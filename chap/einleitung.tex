\chapter{Introduction}
\label{Introduction}

With the growing amount of generated data, the size of files is increasing as well. It is an important topic to deal with the problem of efficient loading and importing, especially in times of big data where file sizes of several gigabytes are common. Also each second of high-performance computing is very valuable and in many cases algorithms can not proceed with an incomplete dataset. This results in long waiting times, while the loading task is running.\\
The data is often stored in graph like networks, where data is represented by vertices and edges. These graphs can consist of billions of small objects, in which case the meta data overhead must be very small to work efficiently. There are many kinds of various graph file formats but this thesis will only focus on the most common and relevant ones, which serve the purpose of exchange the information between two systems.\\\\
This works tries to resolve the problems, which comes with trying to use the resources of a distributed system to accelerate the loading of graphs into memory and manage the chunk distribution in the network. To accomplish this the file format must be parallelizable. This needs to be determined first. The graph files formats, this work deals with, will be evaluated to their parallelizability, their relevance and current state of use. Over the course of time various graph file formats got introduced with many different properties and purposes. They vary from simple edge lists to complex formats in binary or xml format.\\\\
The current state-of-the-art standard is that a parallelizable file format gets used and mapped on a map-reduce framework. Therefor often simple formats are used, like neighbor lists, where the overhead to deal with is very small and one line describes exactly one node and its edges. It's common practice that complex formats get parsed into simpler ones and then read in. This work will try to accomplish a map-reduce-like approach with many more formats, while trying to preserve the performance gained by the resources of a distributed system.
